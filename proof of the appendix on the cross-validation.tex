\documentclass[12pt]{amsart}

\usepackage{setspace,graphicx,srcltx,harvard,threeparttable}
\usepackage[nolists]{endfloat}
\usepackage{float}
\usepackage{harvard}
\usepackage{bm}
\usepackage{caption}
\usepackage{color}
\usepackage{amssymb}%花体字母加粗
\usepackage{mathrsfs}%花体字母
%\usepackage[final]{trackchanges}
%\usepackage{trackchanges}
%\DeclareMathOperator{\sgn}{sgn}
\renewcommand{\subjclassname}{JEL Classification: }
%\renewcommand{\initialsOne}{CP}
%\renewcommand{\initialsTwo}{PD}
%\renewcommand{\initialsThree}{JR}

\newcommand{\bmx}{\bm{x}}
\newcommand{\bmy}{\bm{y}}
\newcommand{\bmw}{\bm{w}}
\newcommand{\bmq}{\bm{q}}
\newcommand{\bmp}{\bm{p}}
\newcommand{\bmz}{\bm{z}}

\newtheorem{theorem}{Theorem}[section]
\newtheorem{assumption}{Assumption}[section]
\newtheorem{lemma}{Lemma}[section]

%% Change the default page sizes.

% \newcommand{\citeaffixed}{XX}

\setlength{\topmargin}{-0.25in} \setlength{\textheight}{8.5in}
\setlength{\oddsidemargin}{.0in} \setlength{\evensidemargin}{.0in}

\setlength{\textwidth}{6.5in} \setlength{\footskip}{.5in}

\title[Appendix: Sketches of the proofs]{Proof of Proposition 1}

\author{Jianhua Zhou}

\author{Subal C.~Kumbhakar}

\author{Christopher F.~Parmeter}

\thanks{All {\tt R} code used in this paper is available upon request. All errors are ours alone.}

\keywords{Partly Linear, Heteroskedasticity, Kernel, Bandwidth.}

\date{\today}

\begin{document}
\section{the main idea about the paper}
the main idea of this paper is to find a surrogate measure of the expected ASE, and then find the optimal bandwidth based on the surrogate measure by discounting the outliers of the data set.\\
So the basic process of the proof is: 1: find the relationship between the RCRVD(h) and the ASE; 2, prove there is no difference about the optimal bandwidth chosen by the RCRVD or the ASE as the $n$ goes to $\infty$. 3. After the proving, find the optimal h from the surrogate measure.\\
\subsection{the proof of the proposition 1}
The Proposition 1:
\begin{align}
RCRVD(h)=c+\dfrac{E\psi^{'}(.)}{2}MASE(h)+o_{p}(n^{-4/5}) \nonumber
\end{align}
$RCRVD(h)$ is the robust cross-validation rule.
Find the sup norms in the one of the interval in the partition $(\zeta_{1}n^{-1/5},\zeta_{2}n^{-1/5})$\\
The \textbf{definition} of the $ RCRVD(h)$
\begin{align}
RCRVD(h)=n^{-1}\sum_{i=1}^{n}\rho(\epsilon_{i}+m_{i}-\widetilde{m}_{-i})
\end{align}
$\rho$ is the function like from $Huber$ function, and $m_{i}$ is the true function and $\widetilde{m}_{-i}$ is the estimated function with leaf-one-out.\\
expand the $\rho$ function by Taylor series at the points $\epsilon_{i},i=1,...,n$.
the Taylor series function:
\begin{align}
\sum^{\infty}_{n=0}\dfrac{f^{n}(a)}{n!}(x-a)^{n}\nonumber
\end{align}
here $a$ is the point the expanded point, $x-a$ is the difference between the $x$ and the expanded point $a$, $f^{n}(a)$ is the order differentiate number.\\
In this case, the function $\rho$ is the function $f$, and the expanded point is the $\epsilon_{i}$, setting the order number as 2, and the difference: 
\begin{align}
\triangle=\epsilon_{i}+m_{i}-\widetilde{m}_{-i}-\epsilon_{i}=m_{i}-\widetilde{m}_{-i}\nonumber
\end{align}
then the expanded function for the function $\rho$ is :
\begin{align}
RCRVD(h)&=n^{-1}\sum^{n}_{i=1}\rho(\epsilon_{i})+D_{1}(h)+D_{2}(h)+D_{3}(h)\nonumber\\
D_{1}(h)&=n^{-1}\sum^{n}_{i=1}(m_{i}-\widetilde{m}_{-i})\psi(\epsilon_{i})\nonumber\\
D_{2}(h)&=n^{-1}\sum^{n}_{i=1}\dfrac{(m_{i}-\widetilde{m}_{-i})^{2}}{2}\psi^{'}(\epsilon_{i})\nonumber\\
D_{3}(h)&=n^{-1}\sum^{n}_{i=1}R_{i}\nonumber
\end{align}
here, $\psi$ is the bounded first derivative of function $\rho$, and the $D_{3}(h)$ is the remainder. Three things to prove:
\begin{align}
D_{1}(h)&=o_{p}(n^{-4/5})\\
D_{2}(h)-\dfrac{E\psi^{'}}{2}MASE(h)&=o_{p}(n^{-4/5})\\
D_{3}(h)&=o_{p}(n^{-4/5})
\end{align}
based on the definition of the robust smoother $\widetilde{m}(.)$ of $m(.)$:
\begin{align}
n^{-1}\sum_{i=1}^{n}\widetilde{\rho}(y_{i}-\widetilde{m}(x_{i}))=min \nonumber
\end{align}
the robust smoother can be defined as the zero:
\begin{align}
\sum_{i=1}^{n}\alpha_{i}(x)\widetilde{\psi}(y_{i}-.)
\end{align}
the Taylor expand for the equation (5) at the point $m(x)$ is :
\begin{align}
\sum_{j=1}^{n}\alpha_{j}(x_{i})\widetilde{\psi}(y_{i}-m_{i})+\sum_{j=1}^{n}\alpha_{j}(x_{i})(m_{i}-\widetilde{m}_{i})\widetilde{\psi}^{'}(y_{i}-m_{i}+\nu_{i})=0
\end{align}
where $|\nu_{i}|<|m_{i}-\widetilde{m}_{i}|$, if there exists a constant $q>0$ such that:
\begin{align}
\sup_{h}|\sum_{j=1}^{n}\alpha_{j}(x_{i})\widetilde{\psi}^{'}(y_{i}-m_{i}+\nu_{i})-q|\rightarrow 0   \quad a.s.\nonumber\\
\Rightarrow q=|\sum_{j=1}^{n}\alpha_{j}(x_{i})\widetilde{\psi}^{'}(y_{i}-m_{i}+\nu_{i})|\nonumber
\end{align}
in the event set $\mathbb{P}=\{inf_{h}\sum_{j=1}^{n}\alpha_{j}(x_{i})\widetilde{\psi}^{'}(y_{i}-m_{i}+\nu_{i})>0\}$, in this set, the 
\begin{align}
q=\sum_{j=1}^{n}\alpha_{j}(x_{i})\widetilde{\psi}^{'}(y_{i}-m_{i}+\nu_{i})
\end{align}
plunge the equation (7) into the equation (6), then 
\begin{align}
\sum_{j=1}^{n}\alpha_{j}(x_{i})\widetilde{\psi}(y_{i}-m_{i})+(m_{i}-\widetilde{m}_{i})q=0\nonumber\\
\Rightarrow m_{i}-\widetilde{m}_{i}=-\dfrac{\sum_{j=1}^{n}\alpha_{j}(x_{i})\widetilde{\psi}(y_{i}-m_{i})}{q}\nonumber
\end{align}
put the equation (19) into the $D_{1}(h)$, then:
\begin{align}
D_{1}(h)=-q^{-1}n^{-1}\sum_{i=1}^{n}\sum_{j\neq i}\alpha_{j}(x_{i})\psi(\epsilon_{j})\psi(\epsilon_{i}).\nonumber\\
|D_{1}(h)|=q^{-1}n^{-1}\sum_{i=1}^{n}\sum_{j\neq i}\alpha_{j}(x_{i})\psi(\epsilon_{j})\psi(\epsilon_{i})
\end{align}
First prove the statement in the equation (9), first we say:
\begin{align}
P(A+B)\leq P(A)+P(B)\nonumber
\end{align}
Assume there are intervals $s_{1}\leq s_{2}\leq...\leq s_{m}$ forming a partition of the interval $(\zeta_{1}n^{-1/5},\zeta_{2}n^{-1/5})$ and the sizes of the intervals $(s_{e-1},s_{e}),e=2,...,m$, are chosen small enough. $s_{e-1}<h_{e}<s_{e}$.
\begin{align}
\textbf{sup}|-D_{1}(h_{e})+D_{1}(s_{e})+D_{1}(h_{e})|\leq \textbf{sup}|D_{1}(h_{e})-D_{1}(s_{e})|+\textbf{sup}|D_{1}(h_{e})| \nonumber\\
P(\sup_{h}|D_{1}(h)|\geq n^{-4/5}\epsilon)\leq P(\sup_{e=2,...,m} \sup_{s_{e-1}<h_{e}<s_{e}}|D_{1}(h_{e})-D_{1}(s_{e})|\geq n^{-4/5}\epsilon/2)\\+P(\sup_{e=2,...,m}|D_{1}(h_{e})|\geq n^{-4/5}\epsilon/2) \nonumber
\end{align}
if we make the interval really small, m goes really big, then the first term of the probabilty is really small, so we only need to consider the second term.\\
Using the following theorem :
\begin{align}
P(X\geq a)\leq \dfrac{E(X^{n})}{a^{n}}\nonumber\\
{\mathbb {P} }{\biggl (}\bigcup _{i}A_{i}{\biggr )}\leq \sum _{i}{\mathbb {P} }(A_{i}).\nonumber
\end{align}
in our case pick up the sup value among the sets is the $\bigcup _{i}A_{i}$, since we have m intervals, then
\begin{align}
P(\sup_{e=2,...,m}|D_{1}(h_{e})|\geq n^{-4/5}\epsilon/2)&\leq m \sup_{e=2,...,m}P(|D_{1}(h_{e})|\geq n^{-4/5}\epsilon/2) \nonumber\\
&\leq m \sup_{e=2,...,m} \dfrac{E(|D_{1}(h_{e})|)^{2k}}{(n^{-4/5}\eta)^{2k})}\\
&=m \sup_{e=2,...,m} E(n^{4/5}\eta^{-1} |D_{1}(h_{e})|)^{2k}, \eta=\epsilon/2 \nonumber
\end{align}
here $k=1,2,,....$
using the Theorem 2 of Whittle P, based on the definition of the $\alpha_{i}$,
\begin{align}
&\alpha_{i}=h^{-1}K\bigg(\dfrac{x-x_{i}}{h}\bigg)\bigg/ \bigg((nh)^{-1}\sum_{i=1}^{n}K\bigg(\dfrac{x-x_{i}}{h}\bigg)\bigg)
\\
&E(n^{4/5}\eta^{-1} |D_{1}(h_{e})|)^{2k} \nonumber\\
\leq & c\bigg[n^{8/5}(\eta qnh_{e})^{-2} \nonumber\\
&\times \sum_{i=1}^{n}\sum_{j\neq i}^{}\bigg\{nhK\bigg(\dfrac{x_{i}-x_{j}}{h_{e}}\bigg)\bigg/ \sum_{l=1}^{n}K\bigg(\dfrac{x_{i}-x_{j}}{h_{e}}\bigg)\bigg\}^{2}\\
&\times E\psi^{2}(\epsilon_{i})E\psi^{2}(\epsilon_{j})\bigg]^{k} \nonumber
\end{align}
here since $K\bigg(\dfrac{x_{i}-x_{j}}{h_{e}}\bigg)>0$ then 
\begin{align}
K\bigg(\dfrac{x_{i}-x_{j}}{h_{e}}\bigg)\bigg/ \sum_{l=1}^{n}K\bigg(\dfrac{x_{i}-x_{j}}{h_{e}}\bigg)<=1\\
\sum_{i=1}^{n}\sum_{j\neq i}^{}\bigg\{K\bigg(\dfrac{x_{i}-x_{j}}{h_{e}}\bigg)\bigg/ \sum_{l=1}^{n}K\bigg(\dfrac{x_{i}-x_{j}}{h_{e}}\bigg)\bigg\}^{2}<=(n*(n-1))<n^2
\end{align}
\textcolor{red}{the equation seems like reduced too much, maybe the multinomial theorem make the equation.}
so the above the equation (32) would be 
\begin{align}
E(n^{4/5}\eta^{-1} |D_{1}(h_{e})|)^{2k}<&c\bigg[n^{(-4+8/5+2)k}h_{e}^{-2k}(E\psi^{2}(\epsilon_{i}))^{2k}\bigg] \nonumber\\
&c\bigg[n^{(-2+8/5)k}h_{e}^{-2k}(E\psi^{2}(\epsilon_{i}))^{2k}\bigg]
\end{align}
based on such statement, 
\begin{align}
m \sup_{e=2,...,m} E(n^{4/5}\eta^{-1} |D_{1}(h_{e})|)^{2k} \nonumber\\
\leq m \sup_{e=2,...,m}  c\bigg[n^{(-2+8/5)k}h_{e}^{-2k}(E\psi^{2}(\epsilon_{i}))^{2k}\bigg]
\end{align}
which goes to zero for fixed $m$,$h_{e}\in [s_{e-1},s_{e}]$ and $n$ going to $\infty$, if $k$ is chosen to be sufficiently large. 
Until here, we have already finished the (A.1)$\blacksquare$
\section{ Proof for the (A.2)}
The logic of this part :
\begin{align}
D_{2}(h)-ED_{2}(h)=o_{p}(n^{-4/5})\\
E(D_{2})=E\psi^{'}(\cdot)MASE(h)/2+o_{p}(n^{-4/5})\\
then\nonumber\\\
D_{2}(h)=E\psi^{'}(\cdot)MASE(h)/2+o_{p}(n^{-4/5})
\end{align}
The Proof:
\\
The definition:
\begin{align*}
D_{2}(h)=n^{-1}\sum_{i=1}^{n}\dfrac{(m_{i}-\widetilde{m}_{-i})^2}{2}\psi^{'}(\epsilon_{i}) \\
ED_{2}(h)=n^{-1}\sum_{i=1}^{n}E\dfrac{(m_{i}-\widetilde{m}_{-i})^2}{2}E\psi^{'}(\epsilon_{i})
\\
or 
D_{2}(h)=n^{-1}\sum_{i=1}^{n}\dfrac{(m_{i}-\widetilde{m}_{i})^2}{2}\psi^{'}(\epsilon_{i}) \\
ED_{2}(h)=n^{-1}\sum_{i=1}^{n}E\dfrac{(m_{i}-\widetilde{m}_{i})^2}{2}E\psi^{'}(\epsilon_{i})
\end{align*}
\begin{align}
D_{2}(h)-ED_{2}(h)&=n^{-1}\sum_{i=1}^{n}\dfrac{(m_{i}-\widetilde{m}_{i})^2}{2}\psi^{'}(\epsilon_{i})-n^{-1}\sum_{i=1}^{n}E\dfrac{(m_{i}-\widetilde{m}_{-i})^2}{2}E\psi^{'}(\epsilon_{i})\nonumber\\
&=\dfrac{1}{2n}\bigg[\sum_{i=1}^{n}((m_{i}-\widetilde{m}_{i})^2-E(m_{i}-\widetilde{m}_{-i})^2+E(m_{i}-\widetilde{m}_{-i})^2)\psi^{'}(\epsilon_{i})\nonumber\\
&-\sum_{i=1}^{n}E(m_{i}-\widetilde{m}_{-i})^2E\psi^{'}(\epsilon_{i})\nonumber\\
&=\dfrac{1}{2n}\bigg[\sum_{i=1}^{n}((m_{i}-\widetilde{m}_{i})^2-E(m_{i}-\widetilde{m}_{-i})^2\psi^{'}(\epsilon)-\sum_{i=1}^{n}E(m_{i}-\widetilde{m}_{-i})^2(\psi^{'}(\epsilon)-E\psi^{'}(\epsilon_{i}))\bigg]\nonumber\\
&=\dfrac{1}{2}\bigg[n^{-1}\sum_{i=1}^{n}W^{i}_{i}(h)+n^{-1}\sum_{i=1}^{n}W_{2}^{i}(h)\bigg]
\end{align}
the next step is to verify the first and second term are uniformly in h to $o_{p}(n^{-4/5})$\\
In our assumption, the $\psi(.)$ is bounded, so for $n^{-1}\sum_{i=1}^{n}W_{1}^{i}$ it is enough just to care about the part $[((m_{i}-\widetilde{m}_{i})^2-E(m_{i}-\widetilde{m}_{-i})^2],i=1,...,n$ are $o_{p}(n^{-4/5})$
\\
The logic of expanding is to open the $sum^2$ term and hope to find the term like $\psi(y_{j}-m_{i})-E\psi(y_{j}-m_{j})$, the expectation of such term is zero.
\begin{align}
\bigg[&\dfrac{\sum_{j\neq i}\alpha_{j}(x_{i})\psi(y_{j}-m_{i})}{q}\bigg]^{2}-E\bigg[\dfrac{\sum_{j\neq i}\alpha_{j}(x_{i})\psi(y_{i}-m_{i})}{q}\bigg]^{2}\nonumber\\
&\bigg \{\sum_{j\neq i} \sum_{k\neq i}\alpha_{j}(x_{i})\alpha_{k}(x_{i})\{\psi(y_{j}-m_{i})-E\psi(y_{j}-m_{i})\}\nonumber\\
&\times\{\psi(y_{k}-m_{i})-E\psi(y_{k}-m_{i})\}\bigg \}\bigg/q^{2}\\
&+\dfrac{\sum_{j\neq i} \sum_{k\neq i}\alpha_{j}(x_{i})\alpha_{k}(x_{i})\{\psi(y_{j}-m_{i})E\psi(y_{k}-m_{i})\}}{q^2}\nonumber\\
&+\dfrac{\sum_{j\neq i} \sum_{k\neq i}\alpha_{j}(x_{i})\alpha_{k}(x_{i})\{\psi(y_{k}-m_{i})E\psi(y_{j}-m_{i})\}}{q^2}\nonumber\\
&-\dfrac{\sum_{j\neq i} \sum_{k\neq i}\alpha_{j}(x_{i})\alpha_{k}(x_{i})\{E\psi(y_{k}-m_{i})E\psi(y_{j}-m_{i})\}}{q^2}\nonumber\\
&-E\bigg[\dfrac{\sum_{j\neq i} \sum_{k\neq i}\alpha_{j}(x_{i})\alpha_{k}(x_{i})\{\psi(y_{j}-m_{i}) \psi(y_{k}-m_{i})\}}{q^2}\bigg]
\end{align}
for the equation (41), if $k=j$ then we have:$E\psi^{2}(y_{j}-m_{i})$, in this logic, then for the equation (41) and (42), we have:
\begin{align}
-\dfrac{\sum_{j\neq i} \alpha^{2}_{j}(x_{i})\{E(\psi(y_{j}-m_{i}))^{2}\}}{q^2}-\dfrac{\sum_{j\neq i} \sum_{k\neq i,k\neq j}\alpha_{j}(x_{i})\alpha_{k}(x_{i})\{E\psi(y_{k}-m_{i})E\psi(y_{j}-m_{i})\}}{q^2}\nonumber\\
-\dfrac{\sum_{j\neq i} \alpha^{2}_{j}(x_{i})\{E\psi^{2}(y_{j}-m_{i})\}}{q^2}-\dfrac{\sum_{j\neq i} \sum_{k\neq i,k\neq j}\alpha_{j}(x_{i})\alpha_{k}(x_{i})\{E\psi(y_{k}-m_{i})E\psi(y_{j}-m_{i})\}}{q^2}\nonumber
\end{align}
so plunge the equation (44), (45) into the process, we got:
\begin{align}
\bigg[&\dfrac{\sum_{j\neq i}\alpha_{j}(x_{i})\psi(y_{j}-m_{i})}{q}\bigg]^{2}-E\bigg[\dfrac{\sum_{j\neq i}\alpha_{j}(x_{i})\psi(y_{i}-m_{i})}{q}\bigg]^{2}\nonumber\\
&=\bigg \{\sum_{j\neq i} \sum_{k\neq i}\alpha_{j}(x_{i})\alpha_{k}(x_{i})\{\psi(y_{j}-m_{i})-E\psi(y_{j}-m_{i})\}\nonumber\\
&\times\{\psi(y_{k}-m_{i})-E\psi(y_{k}-m_{i})\}\bigg \}\bigg/q^{2}\nonumber\\
&+\dfrac{\sum_{j\neq i} \sum_{k\neq i}\alpha_{j}(x_{i})\alpha_{k}(x_{i})\{\psi(y_{j}-m_{i})E\psi(y_{k}-m_{i})\}}{q^2}\nonumber\\
&+\dfrac{\sum_{j\neq i} \sum_{k\neq i}\alpha_{j}(x_{i})\alpha_{k}(x_{i})\{\psi(y_{k}-m_{i})E\psi(y_{j}-m_{i})\}}{q^2}\nonumber\\
&-\dfrac{\sum_{j\neq i} \alpha^{2}_{j}(x_{i})\{(E\psi(y_{j}-m_{i}))^{2}+\{E\psi^{2}(y_{j}-m_{i})\}\}}{q^2}\nonumber\\
&-\dfrac{2\sum_{j\neq i} \sum_{k\neq i,k\neq j}\alpha_{j}(x_{i})\alpha_{k}(x_{i})\{E\psi(y_{k}-m_{i})E\psi(y_{j}-m_{i})\}}{q^2}
\end{align}
Here, we would like to combine equation (49) and (50) and (52),then we would also have:
\begin{align}
-\dfrac{\sum_{j\neq i} \alpha^{2}_{j}(x_{i})\{\{E\psi^{2}(y_{j}-m_{i})\}-(E\psi(y_{j}-m_{i}))^{2}\}}{q^2}\nonumber
\end{align}
then we have:
\begin{align}
&W^{i}_{11}(h)=\bigg \{\sum_{j\neq i} \sum_{k\neq i}\alpha_{j}(x_{i})\alpha_{k}(x_{i})\{\psi(y_{j}-m_{i})-E\psi(y_{j}-m_{i})\}\nonumber\\
&\times\{\psi(y_{k}-m_{i})-E\psi(y_{k}-m_{i})\}\bigg \}\bigg/q^{2}\nonumber\\
&W^{i}_{12}(h)=\dfrac{\sum_{j\neq i} \sum_{k\neq i}\alpha_{j}(x_{i})\alpha_{k}(x_{i})\{\psi(y_{j}-m_{i})-E\psi(y_{j}-m_{i})\}E(\psi(y_{k}-m_{i})}{q^2}\nonumber\\
&W^{i}_{13}(h)=\dfrac{\sum_{j\neq i} \sum_{k\neq i}\alpha_{j}(x_{i})\alpha_{k}(x_{i})\{\psi(y_{k}-m_{i})-E\psi(y_{k}-m_{i})\}E(\psi(y_{j}-m_{i})}{q^2}\nonumber\\
&W^{i}_{14}(h)=-\dfrac{\sum_{j\neq i} \alpha^{2}_{j}(x_{i})\{\{E\psi^{2}(y_{j}-m_{i})\}-(E\psi(y_{j}-m_{i}))^{2}\}}{q^2}
\end{align}
if we set $r_{j}=\psi(y_{j}-m_{i})-E\psi(y_{j}-m_{i})$, then expectation of $r_{j}$ is zero. So compared with the (A.8), the last term $E\psi^{2}(\epsilon_{i})=0^{2k}=1$, then so the first three term we have:
\begin{align}
P(\sup_{h}|W^{i}_{1(L=1,2,3)}|\geq n^{-4/5}\epsilon)\leq cmn^{(-2+8/5)k}h_{e}^{-2k}
\end{align}
for the $w^{i}_{14}$, since the expectation of $\psi$ is limited, then the conclusion is same. All of them would vanish as long as a large enough $k$ is chosen. 
\\
Based on the above, we already prove that:
\begin{align}
|E(m_{i}-\widetilde{m}_{-i})^2-E(m_{i}-\widetilde{m}_{i})^2|=o_{p}(n^{-4/5})\nonumber\\
E(m_{i}-\widetilde{m}_{-i})^{2}=O_{p}(n^{-1}h^{-1}+h^{4})\nonumber\\
\Rightarrow\nonumber\\
P(\sup_{h}|W^{i}_{2}|\geq n^{-4/5}\epsilon)\leq cm(n^{-1}h_{e}^{-1}+h_{e}^{4})n^{(-2+8/5)k}h_{e}^{-2k}
\end{align}
same as above, which means $W^{i}_{2}(h)=o_{p}(n^{-4/5})$.
since $E(D_{2})=E\psi^{'}(.)MASE(h)/2+o_{p}(n^{-4/5})$, then (A.2) follows from the above derivations.
\\
\section{The proof of the Proposition 3}
\subsection{properties about the kernel density function}
Some basic technique about the nonparametric method:
The form of the weight function $\alpha_{i}$ :
\begin{align}
\alpha_{i}(x)=(nh)^{-1}K\bigg(\dfrac{x-x_{i}}{h}\bigg)\bigg/\sum_{i=1}^{n}K\bigg(\dfrac{x-x_{i}}{h}\bigg)=\dfrac{K_{h}(x-x_{i})}{\hat{f}_{h}}\nonumber\\
\hat{f}_{h}=n^{-1}\sum_{i=1}^{n}K_{h}(x-x_{i}),K_{h}(u)=h^{-1}K(u/h)\nonumber
\end{align}
two important constants associated with a kernel function $K(.)$ are its variance $\sigma^{2}_{K}=d_{K}$, and roughness $c_{K}$ which is defined as:
\begin{align}
d_{K}=\int z^{2}K(z)dz, c_{K}=\int K^{2}(z)dz \nonumber
\end{align}
assume :
\begin{align}
y_{i}=m(x_{i})+\epsilon_{i}=m(x)+(m(x_{i}-m(x)))+\epsilon_{i}\nonumber
\end{align}
Then:
\begin{align}
\dfrac{1}{nh}\sum_{i=1}^{n}K\bigg(\dfrac{x_{i}-x}{h}\bigg)y_{i}=&\dfrac{1}{nh}\sum_{i=1}^{n}K\bigg(\dfrac{x_{i}-x}{h}\bigg)\bigg[m(x)+(m(x_{i})-m(x))+\epsilon_{i}\bigg]\nonumber\\
&=\hat{f}(x)m(x)+\dfrac{1}{nh}\sum_{i=1}^{n}K\bigg(\dfrac{x_{i}-x}{h}\bigg)(m(x_{i})-m(x))+\dfrac{1}{nh}\sum_{i=1}^{n}K\bigg(\dfrac{x_{i}-x}{h}\bigg)\epsilon_{i}\nonumber\\
&=\hat{f}(x)m(x)+\hat{m}_{1}(x)+\hat{m}_{2}(x)
\end{align}
which means:
\begin{align}
\hat{m}(x)=m(x)+\dfrac{\hat{m}_{1}}{\hat{f}(x)}+\dfrac{\hat{m}_{2}(x)}{\hat{f}(x)}\nonumber\\
\textbf{bias}=\hat{m}_{1}(x)/\hat{f}(x)+\hat{m}_{2}(x)/\hat{f}(x)
\end{align}
the mean and the variance of the function $m(x)$:
\begin{align}
&E(\epsilon_{i}|x_{i})=0\Rightarrow E(\hat{m}_{2}(x))=0\nonumber\\
Var(\hat{m}_{2}(x))&=\dfrac{1}{nh^{2}}E\bigg(K(\dfrac{x_{i}-x}{h})\epsilon_{i}\bigg)^{2}+n\bigg(E\bigg(K(\dfrac{x_{i}-x}{h})\epsilon_{i}\bigg)\bigg)^{2}\nonumber\\
&=\dfrac{1}{nh^{2}}E\bigg(K(\dfrac{x_{i}-x}{h})\epsilon_{i}\bigg)^{2}\nonumber\\
&=\dfrac{1}{nh^{2}}E\bigg(K^{2}(\dfrac{x_{i}-x}{h})\epsilon^{2}_{i}\bigg)\nonumber\\
&=\dfrac{1}{nh^{2}}E\bigg(K^{2}(\dfrac{x_{i}-x}{h})\sigma^{2}_{x_{i}}\bigg), \textbf{depends on}  E(\epsilon_{i}|x_{i})=0\\
\text{Then}&\nonumber\\
Var(\hat{m}_{2}(x))&=\dfrac{1}{nh^{2}}\int K^{2}(\dfrac{x_{i}-x}{h})\sigma^{2}(z)dzf(z)dz
\end{align}
transfer the variable $\dfrac{z-x}{h}=u$, then $z=hu+x$, then 
\begin{align}
Var(\hat{m}_{2}(x))=&\dfrac{1}{nh^{2}}\int K(u)^2\sigma(x+hu)f(x+hu)dhu\nonumber\\
Var(\hat{m}_{2}(x))=&\dfrac{1}{nh^{2}}\int K(u)^2\sigma(x+hu)f(x+hu)hdu\nonumber\\
Var(\hat{m}_{2}(x))=&\dfrac{1}{nh}\int K(u)^2\sigma(x+hu)f(x+hu)dhu+o(\dfrac{1}{nh})\nonumber\\
=&\dfrac{\sigma^2(x)f(x)}{nh}c_{k}+o(\dfrac{1}{nh})\nonumber
\end{align}
if $h \longrightarrow 0, Nh\longrightarrow\infty$, then applying CLT:
\begin{align}
\sqrt{nh}\hat{m}_{}(x)\underrightarrow{d} N(0,\sigma^{2}(x)f(x)c_{k})
\end{align}
The mean of $m_{1}(x)$ and the variance:
\begin{align}
E(\hat{m}_{1}(x))=\dfrac{1}{h}E[K(\dfrac{x_{i}-x}{h})(m(x_{i})-m(x))]=\dfrac{1}{h}\int K(\dfrac{z-x}{h})(m(z)-m(x))f(z)dz\nonumber
\end{align}
using the same way as above, we have:
\begin{align}
E[\hat{m}_{1}(x)]=\int K(u)(m(x+hu)-m(x))f(x+hu)du \nonumber
\end{align}
Expand $m(x+hu)$ in 2nd-order, and expand the $f(x+hu)$ into 1st order Taylor expansions around $x$, we get:
\begin{align}
E[\hat{m}_{1}(x)]&=\int K(u)(m(x+hu)-m(x))f(x+hu)du\nonumber\\
&=\int K(u)(m(x)+hum^{'}(x)+\dfrac{h^{2}u^{2}}{2}m^{''}(x)-m(x))(f(x)+huf^{'}(x))du\nonumber\\
&=\int K(u)(m(x)+hum^{'}(x)+\dfrac{h^{2}u^{2}}{2}m^{''}(x))(f(x)+huf^{'}(x))du\nonumber\\
&=hm^{'}(x)f(x)\int K(u)udu+h^{2}[\dfrac{1}{2}m^{''}(x)f(x)+m^{'}f^{'}(x)]\int K(u)u^{2}du+o(h^{3})\nonumber\\
\textbf{here} \int k(u)udu=0 \Rightarrow \\
&\approx h^{2}[\dfrac{1}{2}m^{''}f(x)+m^{'}f^{'}(x)]\kappa_{2}=h^{2}\kappa_{2}B(x)f(x)
\end{align}
For the variance part, the logic is almost same, the big difference is that, the expectation is $O(h^{2})$, however the variance is $O(h^{2}/nh)$ which is of smaller order than $O(1/nh)$ which would leads the variance going to zero in the asymptotic way. The following is about that:
\begin{align}
Var(\hat{m}_{1})&=E(\hat{m}^{2}_{1}(x))-E^2(\hat{m}_{1}(x))\nonumber\\
E(\hat{m}^{2}_{1}(x))&=\dfrac{1}{nh^{2}}E[K(\dfrac{x_{i}-x}{h})(m(x_{i})-m(x))]^{2}\nonumber\\
E(\hat{m}^{2}_{1}(x))&=\dfrac{1}{nh}E[k^{2}(\dfrac{x_{i}-x}{h})(m(x_{i})-m(x)))^{2}]\nonumber\\
&=\dfrac{1}{nh}\int k^{2}(u)(m(x+hu)-m(x))^{2}f(x+hu)du\nonumber
\end{align}
do the expedition for $m(x+hu),f(x+hu)$, we get:
\begin{align}
E(\hat{m}^{2}_{1}(x))&=\dfrac{1}{nh}\int k^{2}(u)(m(x+hu)-m(x))^{2}f(x+hu)du\nonumber\\
&=\dfrac{1}{nh}\int k^{2}(u)(m(x)+hum^{'}(x)+\dfrac{h^{2}h^{2}}{2}m^{''}(x)-m(x))^{2}(f(x)+huf^{'}(x))du\nonumber\\
&=\dfrac{1}{nh}\int k^{2}(u)(hum^{'}(x)+\dfrac{h^{2}u^{2}}{2}m^{''}(x))^{2}(f(x)+huf^{'}(x))du\nonumber
\end{align} 
which is $O(\dfrac{h^{2}}{nh})$,when prove that the variance is $O(\dfrac{h^2}{nh})$, then we know the variance would goes to zero in probability.
\begin{align}
\textbf{as } h \longrightarrow 0, \textbf{ and } Nh \longrightarrow \infty , \sqrt{Nh}[\hat{m}_{1}(x)-h^{2}d_{k}B(x)f(x)] \xrightarrow{p} 0
\end{align}
since $\hat{f}(x) \xrightarrow{p} f(x)$
\begin{align}
\sqrt{Nh}[\hat{m}_{1}(x)/\hat{f}(x)-h^{2}d_{k}B(x)] \xrightarrow{p} 0
\end{align}
Then :
\begin{align}
MASE=variance+bias^2&=Var(\hat{m}_{2}(x))+(E(\hat{m}_{1}(x)))^{2}\nonumber\\
&=(nh)^{-1}\sigma^{2}(x)c_{k}/f(x)+h^{4}d^{2}_{k}B(x)^2+o((nh)^{-1}+h^{4})
\end{align}
the optimal bandwidth is that make the MSAE minimal:$h=arg min MASE$, take the derivative, we with respect to $h$, we get:
\begin{align}
-n^{-1}h^{-2}\sigma^{2}(x)c_{k}/f(x)+4h^{3}d^{2}_{k}B^{2}(x)=0\nonumber\\
\Rightarrow h=\bigg[\dfrac{\sigma^{2}(x)c_{k}d^{2}_{k}B^{2}(x)}{4f(x)}\bigg]^{1/5}n^{-1/5}
\end{align}
Based on the result about the h, then assume that:
\begin{align}
D(\epsilon)=\inf_{|u-v|>n^{-1/5}\epsilon}n^{-4/5}|RCRVD(u)-RVRVD(v)|
\end{align}
$n^{-4/5}$ comes from the conclusion from $D_{1},D_{2},D_{3}$, it follows that:
\begin{align}
P(&|h_{RCRDV}-h_{MASE}|>n^{-1/5}\epsilon)\\
&\leq P\bigg(n^{4/5}\sup_{h}|RCRVD(h_{MASE})-RCRVD(h_{RCRVD})|>D(\epsilon)\bigg)\nonumber\\
based on A.13 \nonumber\\
&\leq P\bigg(n^{4/5}\sup_{h}\bigg|RCRVD(h_{MASE})-\dfrac{E\psi^{'}(.)}{2}MASE(h_{MASE})\nonumber\\
&+\dfrac{E\psi^{'}(.)}{2}MASE(h_{MASE})-RCRVD(h_{RCRVD})\bigg|>D(\epsilon)\bigg)
\end{align}
since 
\begin{align}
MASE(h_{MASE})=\min_{h}MASE(h)\leq MASE(h_{RCRVD}) \textbf{ by the definition }\nonumber\\
RCRVD(h_{RCRVD})=\min_{h}RCRVD(h)\leq RCRVD(h_{MASE})\textbf{ by the definition }\nonumber
\end{align}
then can rewrite the function as:
\begin{align}
P(&|h_{RCRDV}-h_{MASE}|>n^{-1/5}\epsilon)\nonumber\\
&\leq P\bigg(n^{4/5}\sup_{h}\bigg|RCRVD(h_{MASE})-\dfrac{E\psi^{'}(.)}{2}MASE(h_{MASE})\bigg|> \dfrac{D(\epsilon)}{2}\nonumber\\
&+P\bigg(n^{4/5}\sup_{h}|\dfrac{E\psi^{'}(.)}{2}MASE(h_{MASE})-RCRVD(h_{RCRVD})\bigg|>\dfrac{D(\epsilon)}{2}\bigg)\nonumber\\
\longrightarrow 0,\nonumber
\end{align}
the rest of the other two are the same.
\section{the proof of the proposition 2}
In the part of proving the proposition 3, we say that the optimal bandwidth would like to choose is that: $ h=\bigg[\dfrac{\sigma^{2}(x)c_{k}d^{2}_{k}B^{2}(x)}{4f(x)}\bigg]^{1/5}n^{-1/5}$, since ASE is $ o((nh)^{-1}+h^{4}) $, then put the optimal bandwidth inside the case, then we have:
\begin{align}
MASE=ASE+o((nn^{-1/5})^{-1})=ASE+n^{-4/5}
\end{align}







\section{the extension proof of the paper si to introduce the c}
A standard normal kernel was used for all smoothing. Robustness in smoothing was chieved by setting $\widetilde{\rho}$ to Huber's.
\begin{align}
\rho(u) \bigg\{
  \begin{tabular}{ll}
   $\dfrac{u^2}{2}$, & if $|u|<c$ , \\
  $c|u|-\dfrac{c^2}{2}$,& if $|u|\geq c$ ,
  \end{tabular}
\end{align}
we can see from the definition that, there is no $c$ inside the function $\rho(u)$, so inside our discussion, the function $\rho$ should related to both of $c, \rho$.
\begin{align}
\rho(u,c) \bigg\{
  \begin{tabular}{ll}
   $\dfrac{u^2}{2}$, & if $|u|<c$ , \\
  $c|u|-\dfrac{c^2}{2}$,& if $|u|\geq c$ ,
  \end{tabular}
\end{align}
\subsection{the meaning of $c$}
the constant $c$ regulates the amount of robustness. The basic logic is that different $c$ value would cut some outliers. For example, in the linear case, if the error term is normal distributed and there is no contamination, the best choice of $c$ is $\infty$. On the other hand, if the error terms follow a heavy-tailed distribution, then $c$ should be chosen to be a small positive value: if the $c$ is small, then the function would always be: $c|u|-\dfrac{c^2}{2}$, which makes the function $\rho$ smaller, and relatively make the sum of error relatively smaller.
\subsection{how to insert the $c$ into the whole process}
inserting the variable $c$ into the process means in order to have a good estimation, we not only need a optimal bandwidth, we also need an optimal constant $c$.
\subsection{the process of find an optimal bandwidth and a optimal constant}
A robust smoother $\tilde{m}(.)$ of $m(.)$ is defined by:
\begin{align}
n^{-1}\sum^{n}_{i=1}\tilde{\rho}\{y_{i}-\tilde{m}(x_{i}),c\}=\text{min}
\end{align}
the first derivative $\tilde{\phi}$ with bounded is :
\begin{align}
\sum^{n}_{i}\alpha_{i}(x)\tilde{\phi}(y_{i}-.,c)
\end{align}
where $\alpha_{i}(x)$ is a weight function based on a kernel $K$ and the bandwidth $h$, the choice of $\tilde{\rho}$ is problem-specific. The weight function $\alpha_{i}(x)$:
\begin{align}
(h)^{-1}K\bigg(\dfrac{x-x_{i}}{h}\bigg)\bigg/\bigg((nh)^{-1}\sum^{n}_{i=1}K\bigg(\dfrac{x-x_{i}}{h}\bigg)\bigg)\nonumber
\end{align}
similarly, the leave-one-out-smoother, $\tilde{m}_{i}$ of $m(.)$ is defined as the zero of :
\begin{align}
\sum_{j\neq i} \alpha_{j}(x)\tilde{\psi}(y_{i}-.,c)\nonumber
\end{align}
Then the robust cross-validation rule with the vary constant $c$ can be defined as:
\begin{align}
\textcolor{red}{RCRVD(h,c)}=n^{-1}\sum_{i=1}^{n}\rho\{y_{i}-\tilde{m}_{-i}(x_{i}),c\})
\end{align}
and 
\begin{align}
ASE(h,c)=n^{-1}\sum_{i=1}^{n}\{\widetilde{m}(x_{i})-m(x_{i})\}^{2}\nonumber\\
\textcolor{red}{MASE(h,c)}=E\{ASE(h,c)|x_{1},...,x_{n}\}\nonumber
\end{align}
the following part is to demonstrate that the equation (69) is a reasonable bandwidth selector.
The basic assumption about the distribution $F(.),m(x),\psi$ are the same as with the original paper:$A1-A8$.
Here, we notice that all the changes by proposing $c$ into the function $\rho$ would happen to the derivatives of the \textcolor{red} {$\rho(h,c)$: $\psi(h,c)$}. So all the assumption from the paper by DenisHeng-Yan Leung about the kernel function would be same:\\
(A1) F(.)  is symmetric about zero\\
(A2) The kernel  K  is symmetric about zero. Furthermore, K is positive, Lipschitz
continuous and satisfies
\begin{align}
\int K(t)dt=1
\end{align}
(A3) The function $m:[0,1]\longrightarrow (c,d)\in R$ is twice differentiable with:
\begin{align}
&\int\{m^{(2)}(x)\}^{2}dx<\infty \text{ and } m^{p}(0)=m^{(p)}(1),p=0,1,2,...,\\
&\text{where } m^{(p)} \text{ denotes the pth derivative of }  \nonumber
\end{align}
(A4) The bandwidth sequence $h$ depends on $n$ and satisfies $h\longrightarrow 0$, $nh\longrightarrow \infty $ as $n \longrightarrow \infty$\\
(A5)The new function \textcolor{red}{$\rho(h,c)$} is a continuous function symmetric about zero and is differentiable everywhere except possibly at a finite number of points.
\\
The next point is find a representation of $\widetilde{m}_{i}$ in a workable form. From the definition if $\widetilde{m}_{i}$
\begin{align}
\sum_{j=1}^{n}\alpha_{j}(x_{i})\widetilde{\psi}(y_{i}-m_{i},c)+\sum_{j=1}^{n}\alpha_{j}(x_{i})(m_{i}-\widetilde{m}_{i})\widetilde{\psi}^{'}(y_{j}-m_{i}+\nu_{i},c)=0
\end{align}
where $|\nu_{i}|<|m_{i}-\widetilde{m}_{i}|$. Therefore, if there exists a constant $q>0$ such that:
\begin{align}
\sup_{h}|\sum_{j=1}^{n}\alpha_{j}(x_{i})\widetilde{\psi}^{'}(y_{i}-m_{i}+\nu_{i},c)-q|\rightarrow 0   \quad a.s.\nonumber\\
\Rightarrow q=|\sum_{j=1}^{n}\alpha_{j}(x_{i})\widetilde{\psi}^{'}(y_{i}-m_{i}+\nu_{i},c)|\nonumber
\end{align}
Just as the Denis paper, on
in the event set $\mathbb{P}=\{inf_{h}\sum_{j=1}^{n}\alpha_{j}(x_{i})\widetilde{\psi}^{'}(y_{i}-m_{i}+\nu_{i},c)>0\}$, in this set, the 
\begin{align}
q=\sum_{j=1}^{n}\alpha_{j}(x_{i})\widetilde{\psi}^{'}(y_{i}-m_{i}+\nu_{i},c)
\end{align}
plunge the equation (7) into the equation (6), then 
\begin{align}
\sum_{j=1}^{n}\alpha_{j}(x_{i})\widetilde{\psi}(y_{i}-m_{i},c)+(m_{i}-\widetilde{m}_{i})q=0\nonumber\\
\Rightarrow m_{i}-\widetilde{m}_{i}=-\dfrac{\sum_{j=1}^{n}\alpha_{j}(x_{i})\widetilde{\psi}(y_{i}-m_{i},c)}{q}\nonumber
\end{align}







 \end{document}



